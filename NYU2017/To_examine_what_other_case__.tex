To examine what other case factors might be the key drivers of delay
we look at case duration for cases with specific characteristics
independently. In \autoref{fig:Violins} we look at the distribution of
case duration (days-to-disposition) through multiple violin
plots. When the data can be split in a binary fashion, a violin plot
allows an intuitive comparison of the two distributions. The different
colors (blue and green) represent case-duration distributions for two
different subsets of the dataset. The distributions are normalized and
smoothed via kernel-density estimate with a Gaussian kernel. The
minimum and maximum values of each distribution reflect the shortest
and longest case in the dataset (and they need not be equal for the
two subsets).  We visualize the distributions of days-to-disposition
in this fashion for the following binary split of the data:
\begin{itemize}
\item gang enhancement \emph{vs} no gang enhancement on the charges,
\item single \emph{vs} multiple charges on the case,
\item single \emph{vs} multiple defendant,
\item guilty plea \emph{vs} any other plea,
\item no contest plea \emph{vs} any other plea,
\item not guilty \emph{vs} any other plea,
\item there was a preliminary hearing \emph{vs} no preliminary hearing on the case,
\item time waived \emph{vs}  time not waived,
\item trial \emph{vs} no trial.
\end{itemize}



More information of these features can be found in
\autoref{tab:Features}. We see that cases where the defendant
initially pleads guilty or no contest to charges are generally
resolved early in the process, while cases where the defendant pleads
not guilty have a flatter distribution, indicating more variability in
the prosecutorial process duration. Cases where the defendant pleads
not guilty are more likely to go to trial, so this is consistent with
what we see in the distribution of durations for cases that have
a preliminary hearing and/or a trial.  The presence of enhancements and
number of defendants are of specific interest as they had been clearly
identified by the SCC DA as possible key contributors to delays in the
prosecutorial process, but, while the first shows more power in the
tail, the presence of more than one defendant on a case or more than
one charge against a defendant do not, somewhat surprisingly, show
significant differences in case duration. However, we emphasize that
the number of cases with multiple defendants and multiple charges is
small, so this difference may not be statistically robust.

