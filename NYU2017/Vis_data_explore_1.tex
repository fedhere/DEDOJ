\subsection{Visual tool to enable data exploration}

While we will perform a statistical analysis of the data, this work
will be generated from a typical data-science approach: finding,
comprehending, merging, and sorting data; and applying statistical
tools and other filters to identify trends in the data. These are not
tasks that are suited for a DA's office, which generally has little
training for this purpose, and has many other important legal tasks to
perform. Therefore, it is desirable to automate much of this process
and provide a means for the prosecutors to engage with their data so
that they can identify trends without advanced data skills.

Even before the final SCC dataset was in our hands, we generated
concepts for the visualization using synthetic datasets. These
datasets were constructed with a small set of features that we
expected would be of interest to the attorneys. This includes the
durations of four phases of prosecution, race and gender, and
age. Although these are only some of the important variables to
consider in our visualization and modeling activities, we chose these
for development purposes so that we could determine how best to handle
arbitrary variables we may want to display. In particular, we have
been able to prototype the ability to filter our data based on binary,
categorical, and continuous variables. All of the engineered features
are enabled in the final version of the dashboard, and new variables
can be easily added on as needed.

The simplest form of this visualization is a stacked horizontal bar
plot (\autoref{fig:sshot}). Each bar represents a category of
comparison that is selected by the user, (e.g., race/ethnicity, age
group, or court category referring to the court of
disposition). Visual comparisons are made via three information
channels for each bar: its location on the $x$-axis, width, and color.

The location of the bar encodes the time for a given phase to commence
relative to the start of some other chosen phase. Location attributes
are most easily compared by a user when they are placed on the same
scale \hyperref[csl:13]{(Munzner 2014}; \hyperref[csl:14]{Wilkinson
  2005)}. Therefore, we provide the ability to choose which phase to
compare against and align the $x$-axis (time) such that the phase
begins at time $t = 0$, and earlier phases are displayed on the
negative portion of the scale. For overall case-duration comparison,
we align to the beginning of the first phase, where the start of each
case is displayed at $t = 0$.
