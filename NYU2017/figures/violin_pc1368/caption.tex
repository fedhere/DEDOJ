Duration of the prosecutorial process to disposition of cases for defendants initially in custody (blue) compared to defendants initially not in custody (green). Significant differences are observed, especially in the time-to-sentence, the distribution of which peaks later and has more power in the tail, and in the post-sentence duration, with an accumulation of defendant continuing to have court dates scheduled years after the beginning of the case. While these events occur after sentence and do not affect the primary metric we are testing (time-to-disposition and particularly when time-to-disposition extends past a year) it may affect the efficiency of the courts and cause delays in other cases. Details of the graphics are as in Figure \ref{fig:Violins}.
\label{fig:ViolinsCompetency}